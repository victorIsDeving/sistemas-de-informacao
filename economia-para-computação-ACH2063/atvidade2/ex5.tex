\section{
    Conclusão
    }

\setlength{\parindent}{4em}
\setlength{\parskip}{0.5em}
\renewcommand{\baselinestretch}{1}

O estudo entregou diversas evidências contrapunham a Teoria Clássica, que era a teoria mais aceita e vigente na época, entregando os princípios básicos da Teoria das Relações Humanas. As conclusões do trabalho, de acordo com Chiavenato \cite{chiavenato2003introduccao}, são as seguintes:

\begin{itemize}
	\item \textbf{O nível de produção é resultante da integração social}: o nível de produção não é determinado pela capacidade física ou fisiológica do empregado (como afirmava a Teoria Clássica), mas por normas sociais e expectativas grupais. É a capacidade social da
	trabalhadora que determina o seu nível de competência e eficiência, e não sua capacidade de executar movimentos eficientes dentro do tempo estabelecido. Quanto maior a integração social no grupo de trabalho, tanto maior a disposição de produzir. Se a empregada apresentar excelentes condições físicas e fisiológicas para o trabalho e não estiver socialmente integrada, sua eficiência sofrerá a influência de seu
	desajuste social.
	
	\item \textbf{Comportamento social dos empregados}: o comportamento do indivíduo se apoia totalmente no grupo. As trabalhadoras não agem ou reagem isoladamente como indivíduos, mas como membras de grupos. A qualquer desvio das normas grupais, a trabalhadora sofre punições sociais ou morais das colegas, no intuito de se ajustar aos padrões do grupo. Enquanto os padrões do grupo permanecerem imutáveis, o indivíduo resistirá às mudanças para não se afastar delas.
	
	\item \textbf{Recompensas e sanções sociais}: o comportamento das trabalhadoras está condicionado a normas e padrões sociais. As operárias que	produziram acima ou abaixo da norma socialmente determinada perderam o respeito e a consideração das colegas. As operárias preferiram produzir menos – e ganhar menos – a pôr em risco suas relações amistosas com as colegas. Cada grupo social desenvolve crenças e expectativas em relação à administração. Essas crenças e expectativas – sejam reais ou imaginárias – influem nas atitudes e nas normas e padrões de comportamento que o grupo define como aceitáveis. As pessoas são avaliadas pelo grupo em relação a essas normas e padrões de comportamento: são bons colegas se seu comportamento se ajusta a suas normas e padrões de comportamento ou são péssimos colegas se o comportamento se afasta delas.
	
	\item \textbf{Grupos informais}: enquanto os clássicos se preocupavam com aspectos formais da organização (como autoridade, responsabilidade, especialização, estudos de tempos e movimentos, princípios gerais de administração, departamentalização, etc.), os autores humanistas se concentravam nos aspectos informais da organização (como grupos informais, comportamento social dos empregados, crenças, atitude e expectativa, motivação, etc.). A empresa passou a ser visualizada como uma organização social composta de grupos sociais
	informais, cuja estrutura nem sempre coincide com a organização formal da empresa. Os grupos informais constituem a organização humana da empresa, muitas vezes em contraposição à organização formal estabelecida pela direção. Os grupos informais definem suas regras de 	comportamento, formas de recompensas ou sanções sociais, objetivos, escala de valores sociais, crenças e expectativas que cada participante vai assimilando e integrando em suas atitudes e comportamento.
	
	\item \textbf{Relações humanas}: no local de trabalho, as pessoas participam de grupos sociais dentro da organização e mantêm-se em uma
	constante interação social. Para explicar o comportamento humano nas organizações, a Teoria das	Relações Humanas passou a estudar essa interação social. As relações humanas são as ações e as atitudes desenvolvidas com os contatos entre pessoas e grupos. Cada pessoa tem uma personalidade própria e	diferenciada que influi no comportamento e nas atitudes das outras com quem mantém contatos e viceversa. As pessoas procuram ajustar-se às demais pessoas e grupos: querem ser compreendidas, aceitas e	participar, no intuito de atender a seus interesses e aspirações. O comportamento humano é influenciado pelas atitudes e normas informais existentes nos grupos dos quais participa. É dentro da organização que surgem as oportunidades de relações humanas, graças ao grande número de grupos e interações
	resultantes. A compreensão das relações humanas permite ao administrador melhores resultados de seus subordinados e a criação de uma atmosfera na qual cada pessoa é encorajada a exprimir-se de forma livre	e sadia.
	
	\item \textbf{Importância do conteúdo do cargo}: a especialização não é a maneira mais eficiente de divisão do trabalho. Embora não tenham se preocupado com esse aspecto, Mayo e seus colaboradores verificaram que a especialização proposta pela Teoria	Clássica não cria a organização mais eficiente. Observaram que os operários trocavam de posição para variar e evitar a monotonia, contrariando a política da empresa. Essas trocas provocavam efeitos negativos na produção, mas elevavam a moral do grupo. O conteúdo e a natureza do trabalho têm influência sobre a moral da trabalhadora. Trabalhos simples e repetitivos tornam-se monótonos e maçantes, afetando negativamente a atitude do trabalhador e reduzindo a sua satisfação e eficiência. De certa forma, a perda de eficiência de produção já aconteceria, mas quando ocorre a troca de funções ela vem com alívio da monotonia e aumento da moral do grupo.
	
	\item \textbf{Ênfase nos aspectos emocionais}: os elementos emocionais não planejados e irracionais do comportamento humano merecem atenção
	especial da Teoria das Relações Humanas. Daí a denominação de sociólogos da organização aos autores humanistas.
\end{itemize}