\section{
    Fases do experimento
    }

\setlength{\parindent}{4em}
\setlength{\parskip}{0.5em}
\renewcommand{\baselinestretch}{1}

De acordo com Chiavenato \cite{chiavenato2003introduccao}, o experimento pode ser dividido em quatro fases distintas, a primeira sobre a iluminação, a segunda de montagem de relés (o componente telefônico que a indústria que comprou a pesquisa produzia), a terceira foi o programa de entrevistas e a quarta, e última, a de análise da organização informal das trabalhadoras. Após as quatro fases o programa foi encerrado por falta de verba.

As fases são descritas em detalhe a seguir:

\begin{enumerate}
	\item \textbf{Fase de Iluminação}: a primeira fase pretendia estudar como a intensidade da iluminação interferia na produtividade do grupo de operárias. Foram criados dois grupos de trabalho para desempenhar as mesmas funções com uma variável diferente: um grupo de observação trabalhava sob intensidade de luz variável, enquanto o grupo de controle tinha intensidade constante. Não foi vista uma correlação direta entre os fatores de construção de relés e a iluminação do ambiente, mas foi observada uma tendência difícil de ser isolada, nas palavras de Chiavenato, "os operários reagiam à experiência de acordo com suas suposições pessoais, ou seja, eles se julgavam na obrigação de produzir mais quando a intensidade de iluminação aumentava e o contrário quando diminuía". Foi reconhecido, então, que havia, sim um fator psicológico que tinha uma influência negativa nas trabalhadoras.
	
	\item \textbf{Fase da Montagem de Relés}: a segunda fase consistiu em separar dois grupos em que o grupo controle trabalharia sob condições constantes (o grupo controle), que consistia em 5 mulheres montando os relés, uma que fornecia as peças, um supervisor e um observador, responsável por observar o trabalho e assegurar o espírito de cooperação entre as mulheres. O grupo experimental, que sofreria mudanças nas condições de trabalho foi dividida em 12 períodos, de acordo com Chiavenato, que são os seguintes:
		\begin{enumerate}
			\item Esse período durou 2 semanas. Foi estabelecida a capacidade produtiva em condições normais de trabalho (2.400 unidades semanais por força) que passou a ser comparada com a dos demais períodos.
			\item Esse período durou 5 semanas. O grupo experimental foi isolado na sala de provas, mantendo-se as condições e o horário de trabalho normais e medindo-se o ritmo de produção. Serviu para verificar o efeito da mudança de local de trabalho.
			\item Nesse período, modificou-se o sistema de pagamento. No grupo de controle havia o pagamento por tarefas em grupo. Os grupos eram numerosos – compostos por mais de cem mulheres; as variações de produção de cada uma eram diluídas na produção e não se refletiam no salário individual. Separou-se o pagamento do grupo experimental e, como ele era pequeno, os esforços individuais
			repercutiam diretamente no salário. Esse período durou 8 semanas. Verificou-se aumento de produção.
			\item Esse período marca o início da introdução de mudanças no trabalho: um intervalo de 5 minutos de descanso no período da manhã e outro igual no período da tarde. Verificou-se novo aumento de	produção.
			\item Nesse período, os intervalos de descanso foram aumentados para 10 minutos cada, verificando-se novo aumento de produção.
			\item Nesse período, introduziu-se três intervalos de 5 minutos na manhã e três à tarde. A produção não	aumentou, havendo queixas quanto à quebra do ritmo de trabalho.
			\item Nesse período, voltou-se a dois intervalos de 10 minutos, em cada período, servindo-se um lanche leve. A produção aumentou novamente.
			\item Nesse período, foram mantidas as mesmas condições do período anterior, e o grupo experimental passou a trabalhar até às 16h30, e não até às 17 horas, como o grupo de controle. Houve acentuado aumento da produção.
			\item Nesse período, o grupo experimental passou a trabalhar até às 16 horas. A produção permaneceu estacionária.
			\item Nesse período, o grupo experimental voltou a trabalhar até às 17 horas, como no 7º período. A	produção aumentou bastante.
			\item Nesse período, estabeleceu-se a semana de 5 dias, com sábado livre. A produção diária do grupo experimental continuou a subir.
			\item Nesse período, voltou-se às mesmas condições do 3º período, tirando-se todos os benefícios dados, com o assentimento das operárias. Esse período, último e decisivo, durou 12 semanas. Inesperadamente, a produção atingiu um índice jamais alcançado anteriormente (3 mil unidades semanais por operária).
		\end{enumerate}
	
	\item \textbf{Fase das Entrevistas}: essa fase consistiu em ouvir as operárias sobre suas opiniões sobre o trabalho e o tratamento que recebiam. O programa foi um sucesso e tentaram aplicar para toda a empresa, todos seus empregados seriam entrevistados anualmente, adotando-se uma técnica em que o entrevistador não interferia durante as falas dos empregados nem tentava impor um roteiro de entrevista, o empregado tinha a liberdade de falar sobre o que achava importante. Essa fase revelou um laço de lealdade entre as operárias, que os pesquisadores chamaram de Organização Informal, afim de se protegerem contra o que consideravam um ataque da administração. Para estudar esse fenômeno criaram a quarta fase do experimento.
	
	\item \textbf{Fase da Análise da Organização Informal}: essa fase consistiu numa mistura da segunda e da terceira fase do experimento. Um grupo de operárias iria trabalhar numa sala separada mas com condições idênticas às do departamento, com um observador dentro da sala e um entrevistador do lado de fora. O sistema de pagamento, de acordo com Chiavenato, era o seguinte:
		\begin{quote}
			"O sistema de pagamento era baseado na produção do grupo, havendo um salário-hora com base em fatores e um salário mínimo horário, para o caso de interrupções na produção. Os salários só podiam ser maiores se a produção total aumentasse."
		\end{quote}
	Foram percebidas diversas artimanhas na parte dos operários, como diminuir o ritmo de produção após alcançar o que era consideravam a sua produção normal diária, consideravam um "delator" quem prejudicava uma companheira e até punições simbólicas para quem produzia mais rápido e não buscava "estabilizar" sua produção, além de desenvolverem uma "uniformidade de sentimentos" e "solidariedade grupal".
	
\end{enumerate}
	