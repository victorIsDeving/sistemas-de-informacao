\section{
	Introdução
}

\setlength{\parindent}{4em}
\setlength{\parskip}{0.5em}
\renewcommand{\baselinestretch}{1}

O Experimento de Hawthorne foi um estudo prático que relaciona a produtividade e diversas condições de trabalho, realizado entre 1924 e 1932 \cite{chiavenato2003introduccao}. A pesquisa foi encomendada por uma fábrica da \textit{Western Electric Company}, uma empresa que montava componentes de telefones, localizada no bairro de Hawthorne, em Chicago-IL, e, inicialmente, ela buscava correlacionar a influência da iluminação do ambiente com a produtividade das mulheres que trabalhavam na empresa \cite{mayo2004human}.

A empresa contratou uma equipe liderada pelo psicólogo e sociólogo Elton Mayo, à época professor na Harvard Business School, e que já tinha experiência trabalhando com pesquisa em ambientes industriais e com resultados muito positivos, numa empresta têxtil que queria reduzir a rotatividade de pessoal. Mayo introduziu um intervalo de descanso, delegou aos operários a decisão sobre horários de produção e contratou uma enfermeira. Em pouco tempo, emergiu um espírito de grupo, a produção aumentou e a rotatividade do pessoal diminuiu \cite{chiavenato2003introduccao}.

O resultado da influência da iluminação a \textit{Western Electric Company} foi inconclusivo, porém, Mayo interpretou que poderiam ter mais informações a serem descobertas sobre fadiga e monotonia em ambientes industriais, que eram temas contemporâneos importantes de discussão. Com isso, o estudo cresceu e teve diversas fases de experimentos diferentes realizados.

O experimento é amplamente estudado até hoje e é responsável por criar a Teoria das Relações Humanas, uma nova escola de estudos de administração, a Escola Humanística de Administração, movimento que se opôs à teoria vigente à época, a Teoria Clássica da Administração.

