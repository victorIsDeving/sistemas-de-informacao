\section{
    Terceira Onda: A Sociedade da Informação ou Pós-Industrial 
    }

\setlength{\parindent}{4em}
\setlength{\parskip}{0.5em}
\renewcommand{\baselinestretch}{1}

A terceira onda tira o foco da produção mecânica em massa de bens de consumo e foca na tecnologia da informação. Toffler argumenta que a tecnologia da informação é uma força transformadora fundamental que redefiniu e continuará a redefinir a maneira como as pessoas vivem, trabalham e interagem umas com as outras.

A economia é impulsionada pela informação, com a tecnologia desempenhando um papel central na produção e distribuição de bens e serviços. Há uma ênfase na criatividade, inovação e conhecimento como recursos-chave. A sociedade se torna mais conectada globalmente, com mudanças rápidas e disruptivas sendo comuns. As grandes mudanças são as seguintes:

\begin{itemize}
	\item \textbf{Progresso Tecnológico}: computadores, internet, telecomunicações e outras tecnologias digitais transformaram radicalmente a forma como as informações são criadas, armazenadas, compartilhadas e acessadas.
	\item \textbf{Impacto na Economia}: a economia passa a se basear no conhecimento e na informação. A tecnologia da informação desempenha um papel central na produção e distribuição de bens e serviços, e empresas de tecnologia, como Google, Apple e Amazon, tornaram-se dominantes em muitos setores da economia, desde consumo até entretenimento.
	\item \textbf{Mudanças Sociais}: a globalização e a conectividade digital tornaram o mundo mais interconectado do que nunca, permitindo a colaboração e a comunicação em escala global. O trabalho remoto e a economia do gig foram facilitados pela tecnologia da informação, mudando a natureza e a forma com que os seres humanos se relacionam com o trabalho.
	\item \textbf{Desafios e oportunidades}: houve um aumento na eficiência e na produtividade em muitos setores da economia, bem como uma maior acessibilidade à informação e ao conhecimento, porém, também surgem preocupações sobre privacidade, segurança cibernética, desigualdade digital e o impacto da automação na força de trabalho.
\end{itemize}

Antes da terceira onda, o perfil do trabalhador estava centrado principalmente na indústria e na produção em massa. Os trabalhadores industriais eram empregados em fábricas e indústrias, realizando tarefas específicas dentro de uma linha de produção altamente estruturada. As habilidades exigidas dos trabalhadores eram geralmente relacionadas à operação de máquinas e equipamentos industriais, bem como à capacidade de seguir instruções precisas e trabalhar em equipe. Havia uma alta especificação para trabalhos manuais, mas não necessariamente com alto grau de educação e treinamento, e pouca necessidade de adaptação ou flexibilização no exercer das atividades diárias.

O trabalhador da terceira onda é frequentemente descrito como um profissional do conhecimento ou trabalhador do setor de serviços. São pessoas que lidam com informações, tecnologia e comunicação, e muitas vezes têm habilidades especializadas em campos como programação de computadores, design gráfico, marketing digital, entre outros. Flexibilidade, adaptabilidade e habilidades de aprendizado contínuo são altamente valorizadas nessa era. Torna-se necessária a alfabetização digital, programação de computadores, análise de dados, habilidades de comunicação online e capacidade de adaptação a novas tecnologias e ferramentas digitais.

O foco é crescente em habilidades digitais e tecnológicas. Programas de educação superior e treinamento profissional oferecem cursos e certificações em áreas como ciência da computação, análise de dados, marketing digital e desenvolvimento de aplicativos, para preparar os trabalhadores para as demandas do mercado de trabalho digital.

