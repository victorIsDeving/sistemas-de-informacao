\section{
    Primeira Onda: A Sociedade Agrícola
    }

\setlength{\parindent}{4em}
\setlength{\parskip}{0.5em}
\renewcommand{\baselinestretch}{1}

Antes dessa onda a sociedade vivia a era paleolítica, vivendo da caça e coleta e com assentos nômades. Os grupos era pequenos, muitas vezes familiares, e dependiam exclusivamente do que o ambiente em que eles se assentavam tinha para oferecer, ao esgotar os recursos na região de um assentamento eles se mudavam para outra região. O deslocamento constante era vital para a sobreviv~encia continua do grupo.

As ferramentas disponíveis eram principalmente de pedra lascada, eram usadas para todo tipo de atividade, desde a caça e a preparação de alimentos até atividades cotidianas. Havia uma estratificação social muito simples e uma divisão igualitária entre os membros do grupo.

A primeira onda ocorre, principalmente, com a descoberta da agricultura, daí o nome da onda ser "a sociedade agrícola". Os impactos dessa invenção, nos termos das ondas do Toffler, são as seguintes:

\begin{itemize}
	\item \textbf{Progresso Tecnológico}: as comunidades agrícolas começaram a desenvolver técnicas de cultivo, domesticação de animais e métodos de armazenamento de alimentos mais eficientes.
	\item \textbf{Impacto na Economia}: a agricultura permitiu uma produção excedente de alimentos, o que possibilitou o desenvolvimento de economias mais complexas e diversificadas. O comércio de alimentos e produtos agrícolas tornou-se uma parte importante da economia, levando ao surgimento de mercados e sistemas de troca.
	\item \textbf{Mudanças Sociais}: começaram a surgir divisões de trabalho mais especializadas, com algumas pessoas se dedicando à agricultura enquanto outras desempenhavam funções relacionadas à religião, liderança, artesanato, entre outros. A formação de assentamentos permanentes também levou ao desenvolvimento de sistemas de governo mais complexos e hierárquicos.
	\item \textbf{Desafios e oportunidades}: a agricultura permitiu o crescimento populacional e o estabelecimento de comunidades mais desenvolvidas, que, consequentemente, são maiores e mais complexas de administrar. Além de que a dependência da agricultura tornou as comunidades mais vulneráveis a fatores como secas, pragas e desastres naturais, e levou à necessidade de gerenciar de forma mais eficaz os recursos naturais disponíveis.
\end{itemize}

Temos, então, antes da primeira onda um perfil de trabalhador bastante genérico: todos desempenhavam as mesmas funções de caça, coleta e pesca, em maior ou menor grau. Seus conhecimentos eram, também, muito atrelados à região em que eles ocupavam no momento.

Com a primeira onda e o desenvolvimento da agricultura, o perfil de trabalhador disponível aumentou e ocorreu uma especialização aos diversos trabalhos disponíveis. Agora teríamos uma pessoa especializada na agricultura, outra dedicada ao artesanato, outra a domesticação de animais, etc. Nasce uma divisão clara do trabalho, com papéis definidos por gênero e habilidades específicas, muitas vezes que são transmitidas de geração em geração dentro de uma família.