\section{
	Introdução
}

\setlength{\parindent}{4em}
\setlength{\parskip}{0.5em}
\renewcommand{\baselinestretch}{1}

Alvin Toffler foi um escritor, futurista e sociólogo norte-americano, nascido em 4 de outubro de 1928 e falecido em 27 de junho de 2016. Ele é mais conhecido por seus trabalhos sobre mudança social, tecnologia e suas implicações futuras para a sociedade. Toffler foi um dos mais proeminentes futuristas do século XX e seus livros influenciaram profundamente o pensamento contemporâneo sobre esses assuntos.

Toffler formou-se em jornalismo pela Universidade de Nova York e trabalhou como repórter para várias publicações, incluindo a revista Fortune. Sua carreira como escritor começou na década de 1960, quando ele começou a escrever sobre temas relacionados à mudança social e tecnológica.

Seus principais livros são \textit{Future Shock}, de 1970, \textit{The Third Wave}, de 1980, e \textit{Powershift: Knowledge, Wealth, and Violence at the Edge of the 21st Century}, de 1990. Todos abordam, em diferentes níveis e focos, temas semelhantes relacionados à mudança social e tecnológica. O objetivo desse trabalho é fazer uma resenha e análise o segundo livro dessa "trilogia", o \textit{The Third Wave}, de 1980 \cite{Toffler1980}.

Neste livro, talvez o mais famoso e lido de Toffler, ele argumenta que a sociedade da época em que o livro foi escrito, está passando por uma terceira grande onda de mudança, após a era agrícola e a era industrial. Ele prevê a ascensão da tecnologia da informação, a globalização e a importância crescente do conhecimento na economia moderna, ascensão, esta, que já tinha se iniciado e estava nos seus primórdios. 

A grande metáfora do livro é a da colisão de ondas de mudança. A era da tecnologia e da informação seria essa "terceira onda", dado que a primeira foi a da era agrícola, e a segunda foi a da era industrial. As ondas possuem características semelhantes entre si, compartilhando um padrão geral de mudanças que representa estágios do desenvolvimento na história humana. Algumas das semelhanças entre as ondas são as seguintes:

\begin{itemize}
	\item \textbf{Progresso tecnológico}: cada onda implica em um progresso tecnológico significativo que é o motor de mudanças.
	\item \textbf{Impacto na economia}: cada onda insere elementos que mudam fundamentalmente a economia de sua época.
	\item \textbf{Mudança Social}: cada onda transformou a estrutura da sociedade e a forma com que os seres humanos viviam as suas vidas e se relacionavam com o trabalho.
	\item \textbf{Desafios e oportunidades}: cada onda tem um conjunto muito próprio de desafios a serem resolvidos e de oprtunidades que são oferecidas.
\end{itemize}

