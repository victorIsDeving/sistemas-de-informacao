\section{
    Segunda Onda: A Sociedade Industrial
    }

\setlength{\parindent}{4em}
\setlength{\parskip}{0.5em}
\renewcommand{\baselinestretch}{1}

A sociedade baseada na agricultura persevera por muitos anos e assim se desenvolve. As atividades agrícolas são o principal pilar econômico, configurando uma economia agrária de produção de alimentos e produtos derivados. Toda essa produção era descentralizada, ocorrendo nas pequenas fazendas e oficinas artesanais, centralizadas em famílias e comunidades e focados em consumo local, com muitas limitações e dificuldades para o comércio. A maior parte dessa produção era manual, com algum uso de força de animais domesticados para ajudar com a produção e transporte.

A segunda onda é marcada pelo uso de máquinas, daí o nome de "sociedade industrial". Os impactos dessa invenção são muito profundos e importantes para o desenvolvimento humano a partir de então, são mudanças que não apenas transformam a forma como os bens eram produzidos e distribuídos, mas também influencia profundamente a estrutura social, a organização política e o desenvolvimento econômico das sociedades do período. As grandes mudanças que aconteceram são as seguintes:

\begin{itemize}
	\item \textbf{Progresso Tecnológico}: houve uma ampla adoção de máquinas e equipamentos movidos a vapor, com o surgimento de fábricas e linhas de produção, bem como o desenvolvimento de novas tecnologias e métodos de produção em massa.
	\item \textbf{Impacto na Economia}: a industrialização em larga escala levou ao crescimento do capitalismo industrial. Houve um aumento dramático na produção de bens e uma expansão dos mercados globais.
	\item \textbf{Mudanças Sociais}: surge uma classe trabalhadora industrial, composta por pessoas que trabalhavam em fábricas e indústrias. Houve também uma urbanização em massa com um rápido crescimento populacionas, com um êxodo das áreas rurais para as cidades em busca de emprego nas fábricas.
	\item \textbf{Desafios e oportunidades}: houve um aumento na produtividade e no padrão de vida para muitas pessoas, com a criação de empregos em fábricas e indústrias, porém, as condições de trabalho eram precárias e pioraram progressivamente, além da  exploração da mão de obra, incluindo a infantil, e desigualdades sociais crescentes.
\end{itemize}

Enquanto os trabalhadores antes da segunda onda estavam envolvidos principalmente em atividades relacionadas à produção agrícola e ao comércio, os trabalhadores após a segunda onda estavam envolvidos em uma variedade de ocupações industriais, muitas das quais estavam relacionadas à produção em massa de bens de consumo.

Surge, então, uma classe trabalhadora industrial, composta por pessoas que trabalhavam em fábricas e indústrias. Os trabalhadores industriais eram geralmente empregados em operações de fabricação em larga escala, realizando tarefas repetitivas e específicas dentro de um ambiente de produção altamente organizado. Essas pessoas são o perfil clássico do trabalhador da época, com habilidades relacionadas à operação de máquinas e equipamentos industriais, bem como habilidades específicas relacionadas à linha de produção em que estavam envolvidos. Isso incluía o manuseio de máquinas complexas, o controle de qualidade, a montagem de produtos e outras tarefas relacionadas à produção industrial. Além disso, os trabalhadores também precisavam ser capazes de seguir instruções precisas e trabalhar em equipe dentro de um ambiente altamente estruturado e hierárquico.

Muito do ambiente de trabalho da época e das funções desempenhadas pelos trabalhadores podem ser vistas no filme "Tempos Modernos", de 1936, do Charles Chaplin \cite{Chaplin1936}. O filme também dá um panorama dos desafios desse modo frenético de produção, que tem consequências para a saúde dos trabalhadores da época.