\section{Metodologia}
A metodologia científica do presente projeto de pesquisa se centrará em um conjunto de procedimentos qualitativos e quantitativos, técnicas e ferramentas para a exploração do problema, das hipóteses e dos objetivos previstos. 

Pretende-se verificar se a gestão das políticas de mobilidade urbana influencia a segregação socioespacial das camadas de baixa renda na Região Metropolitana de São Paulo, através da análise da desigualdade de investimentos públicos no sistema viário entre os centros e periferias; das dificuldades de acesso aos serviços e empregos que se concentram nos centros urbanos, além dos impactos desses fatores na qualidade de vida das populações da periferia.

Para que os resultados possam ser alcançados, será realizada uma pesquisa bibliográfica baseada em materiais publicados em livros, periódicos, artigos, documentos, jornais e repositórios públicos como IPEA e IBGE. Além disso, em conjunto, haverá a pesquisa em forma de estudo de casos, através de materiais também já publicados, para o aprofundamento técnico de conceitos ou objetos relevantes para a pesquisa que surgirem no decorrer dos trabalhos. 

Os dados serão coletados pela medição de opinião, seletiva, crítica e analítica, advinda das pesquisas bibliográficas. Após a coleta, os dados serão classificados e organizados em gráficos, tabelas ou quadros.

Com os dados organizados será observado se existem fortes relações, tanto qualitativas quanto quantitativas, que viabilizem a veracidade das hipóteses para a existência do problema proposto pela pesquisa.
